\documentclass[11pt]{article}

\usepackage{amsmath}
\usepackage{hyperref}
\usepackage{xepersian}

\settextfont{B Nazanin}
\setlatintextfont{Times New Roman}

\begin{document}
\begin{titlepage}
	\centering
	\emph{به نام خدا}\par
	\vspace{.5cm}
\rule{\textwidth}{1pt}
	\vspace{.25cm}
	\begin{latin}
		{\huge\bfseries Resource Provisioning in Software Defined Networking Using Network Function Virtualization} \par
	\end{latin}
	\vspace{.5cm}
	\emph{واژه‌های کلیدی:}
	\vspace{.1cm}
	\begin{latin}
		\emph{Software Defined Networking, Resource Provisioning, Distributed Computing, Network Function Virtualization}
	\end{latin}
	\emph{پرهام الوانی (۹۲۳۱۰۵۸)}
\end{titlepage}
\begin{abstract}
	در این مقاله قصد داریم برای بهبود تخصیص منابع در شبکه‌های نرم‌افزار بنیان از مجازی سازی توابع شبکه استفاده کنیم، در این روش تمامی قسمت‌های شبکه به صورت نرم‌افزاری پیاده‌سازی میشوند و برای بهبود زمان لازم برای بررسی بسته‌ها از کتابخانه‌هایی مانند \lr{DPDK} استفاده میکنیم.
\end{abstract}
\section{مقدمه}

\par
ایده شبکه‌های قابل برنامه ریزی به تازگی با توجه به ظهور \lr{SDN} شتاب قابل توجهی گرفته است.
\par
 \lr{SDN} وعده داده است که به صورت چشمگیری مدیریت شبکه را آسان کند و همچنین با استفاده از قابلیت برنامه نویسی شبکه، امکان پیاده سازی ایده‌های جدید در شبکه به سادگی را بوجود آورد.
\par
شبکه‌های نرم افزار بنیان با وجود اینکه مدت زمان زیادی نیست که مطرح شده‌اند ولی در صعنت نیز جای خود را پیدا کرده‌اند و روز به روز گسترده‌تر می‌شوند. چالش‌های بسیاری برای شبکه‌های نرم افزار بنیان وجود دارد که یکی از آن‌ها نحوه تخصیص و آماده سازی منابع در این شبکه‌هاست.
\par
مساله تخصیص منابع در شبکه‌های نرم افزار بنیان یکی از مهمترین چالش‌های این زمینه تحقیقاتی می‌باشد. پلتفرم‌ها و بستر‌های نرم افزاری مختلفی در این زمینه معرفی شده‌اند که در این تحقیق قصد داریم این بستر‌ها و پلتفرم‌ها را بررسی و مقایسه نماییم و نشان دهیم که ایده‌ی مجازی‌سازی توابع شبکه در این امر چه تاثیری دارد. 
\par
پیش از بررسی راه‌کار مورد نظرمان یعنی مجازی‌سازی توابع شبکه، به بررسی سایر راه‌کار‌های موجود می‌پردازیم.
\section{کارهای مرتبط}
\subsection{\lr{Devoflow}}

\par
 \lr{Devoflow}یک افزونه‌ای است که به \lr{OpenFlow} اضافه شده و در شبکه‌های بزرگ با حجم ترافیک بالا استفاده می‌شود.
\par
در این روش برای کاهش سربار \lr{flow-base switching} از دو مکانیزم به نام‌های \lr{Rule cloning} و \lr{Local actions} استفاده می‌شود.
در مدل استاندارد پروتکل \lr{OpenFlow} اگر از قوانینی تحت عنوان \lr{wildcard} استفاده کنیم تا تقاضاها به کنترلر را کاهش دهیم آنگاه تمام بسته‌هایی که با این قوانین هماهنگ هستند تحت عنوان یک \lr{Flow} طبقه بندی می‌شوند. در مکانیزم اول \lr{devoflow} بیان می‌شود که می‌توان در صورت نیاز قوانین \lr{wildcard} را با پرچم \lr{CLONE} مشخص کرده و به این ترتیب آنها را وادار کرد بسته‌های هماهنگ با این قوانین را به \lr{Flow} های مختلف تقسیم کنند. برای این امر سوئیچ یک قانون جدید با مشخصات بسته ورودی هماهنگ با قانون \lr{wildcard} ایجاد میکند و به این ترتیب سایر بسته‌های این \lr{Flow} با قانون جدید مطابقت پیدا میکنند. 
مکانیزم دوم به زبان ساده بیان میکند که برای کاهش تقاضاها به کنترلر می‌توان تعدادی از تصمیم‌ها را در سمت سوئیچ گرفته و سربار  کنترلر را کاهش داد. 
این روش به اینصورت عمل می‌کند که اگر \lr{Flow} ورودی کوچک باشد از \lr{Devoflow} استفاده می‌شود و اگر \lr{Flow} بزرگ باشد از کنترلر استفاده می‌کند که این امر باعث میشود جریان‌های کوچک حافظه‌ای را اشغال نکنند و فضا را در اختیار جریان‌های بزرگتر قرار دهد.


\begin{thebibliography}{9}
\bibitem{morreale15}
	\begin{latin}
		Patricia A. Morreale,
		\emph{Software Defined Networking: Design and Deployment},
		6000 Broken Sound Parkway NW,
		CRC Press,
		2015.
	\end{latin}
\bibitem{wood14}
	\begin{latin}
		Timothy Wood,
		\emph{NetVM: High Performance and Flexible Networking
Using Virtualization on Commodity Platforms },
		11th USENIX Symposium on Networked Systems Design and Implementation
  		2014.
	\end{latin}
\bibitem{eric15}
	\begin{latin}
		Eric Keller,
		\emph{Stateless Network Functions},
		HotMiddlebox’15, August 17-21, 2015, London, United Kingdom.
	\end{latin}
\bibitem{fei14}
	\begin{latin}
		H. Fei, H. Qi and B. Ke,
		\emph{A Survey on Software-Defined Network and OpenFlow: From Concept to Implementation},
		IEEE Communication Surveys \& Tutorials, vol. 16, no. 4,
 		2014.
	\end{latin}
\bibitem{nadeau13}
	\begin{latin}
		Thomas D. Nadeau, Ken Gray,
		\emph{SDN: Software Defined Networks},
		O'Reilly Media,
		2013.
	\end{latin}
\end{thebibliography}
\end{document}

