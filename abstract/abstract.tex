\documentclass[11pt]{article}

\usepackage{amsmath}
\usepackage{hyperref}
\usepackage{xepersian}

\settextfont{B Nazanin}
\setlatintextfont{Times New Roman}

\begin{document}
\begin{titlepage}
	\centering
	\emph{به نام خدا}\par
	\vspace{.5cm}
	\emph{ارائه چکیده تحقیق}\par
	\vspace{.5cm}
	\emph{پرهام الوانی ۹۲۳۱۰۵۸}\par
	\vspace{.25cm}
	\rule{\textwidth}{1pt}
	\vspace{.25cm}
	\begin{latin}
		{\huge\bfseries Resource Provisioning in Software Defined Networking}\par
	\end{latin}
\end{titlepage}
\begin{abstract}
	ایده شبکه‌های قابل برنامه ریزی به تازگی با توجه به ظهور \lr{SDN} شتاب قابل توجهی گرفتهاست.
	\lr{SDN} وعده داده است که به صورت چشمگیری مدیریت شبکه را آسان کند و همچنین با استفاده از قابلیت برنامه نویسی شبکه، امکان پیاده سازی ایده‌های جدید در شبکه به سادگی را بوجود آورد.
شبکه‌های نرم افزار بنیان با وجود اینکه مدت زمان زیادی نیست که مطرح شده‌اند ولی در صعنت نیز جای خود را پیدا کرده‌اند و روز به روز گسترده‌تر می‌شوند. چالش‌های بسیاری برای شبکه‌های نرم افزار بنیان وجود دارد که یکی از آن‌ها نحوه تخصیص و آماده سازی منابع در این شبکه‌هاست.
\end{abstract}
\end{document}

